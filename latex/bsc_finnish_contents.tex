\chapter*{Ohjelmiston arkkitehtuurisuunnittelu ketterissä menetelmissä}

Ohjelmiston arkkitehtuuri on IEEE:n standardin~\citep{jen2000working} mukaan ohjelmiston perusorganisaatio, joka sisältää järjestelmän osat, niiden keskenäiset suhteet ja suhteet ympäristöön. Määritelmän mukaan arkkitehtuuri ohjaa järjestelmän suunnittelua ja evoluutiota. 

Rational Unified Process~\citep{kruchten2004rational} määrittelee ohjelmistoarkkitehtuurin joukkona perustavanlaatuisia päätöksiä koskien järjestelmän organisaatiota, rakenteellisia elementtejä, rajapintoja, alijärjestelmiksi jakautumista, arkkitehtuurillista tyyliä ja niin edelleen. RUP:in mukaan arkkitehtuurissa ei olla kiinnostuneita pelkästään rakenteesta ja toiminnasta, mutta myös sen käyttökohteesta, suorituskyvystä, joustavuudesta, uudelleenkäytettävyydestä ja ymmärrettävyydestä. Arkkitehtuurissa pitäisi ottaa huomioon taloudelliset ja teknologiset rajoitteet.

Nykyään suosioon nousseita ketteriä menetelmiä yhdistää iteratiivinen ja inkrementaalinen elinkaari, pienissä erissä tapahtuvat julkaisut, yhtenäinen tiimi sekä ominaisuus- tai tuotebacklogiin perustuva julkaisusuunnitelma. Olennaista on myös pyrkiminen prosessin yksinkertaisuuteen ja sopeutuminen vaatimusten muutoksiin. Eri ketterien menetelmien kuvaukset ei tyypillisesti kerro mitään arkkitehtuurin suunnittelusta.~\citep{abrahamsson2010agility}

Perinteisessä ohjelmistotuotannossa, kuten vesiputousmallissa, ohjelmistoon liittyvä suunnittelu tapahtuu tyypillisesti ennen varsinaista sovelluksen toteutusvaihetta.
Tämä Big Design Up-Front (BDUF) -käytäntö yhdistetään myös ohjelmiston arkkitehtuurin suunnitteluun.~\citep{abrahamsson2010agility}

Ketterän ohjelmistokehityksen periaatteet ovat epäsuhdassa arkkitehtuuriperustaisen ohjelmistosuunnittelun kanssa. Ketterän manifestin~\citep{fowler2001agile} mukaan  muutokseen reagoimista pidetään tärkeämpänä kuin tarkkojen suunnitelmien noudattamista. Halutaan minimoida kaikki turha työ, sillä arkkitehtuuriin saatetaan joutua tekemään muutoksia ohjelmiston elinkaaren aikana. Ihanteena onkin, että päätökset pyritään tekemään mahdollisimman myöhäisessä vaiheessa: mitä myöhemmin päätökset tehdään, sitä enemmän tietoa on hyödynnettävissä päätösten tueksi. Lisäksi asiakkaalle arvoa tuottava ohjelmiston ominaisuuksien nopea tuottaminen nähdään tärkempänä asiana kuin aikaa vievä tarkka suunnittelu ja dokumentointi. 

Artikkelin Agility and architecture: Can they coexist?~\citep{abrahamsson2010agility} mukaan suurin vastakkainasettelu sijaitsee ketterän kehityksen sopeutumismentaliteetin ja perinteisen suunnittelun ennakoinnin välillä. Ketterässä kehityksessä liian tarkkaa etukäteissuunnittelua vältetään. Arkkitehtuurisuunnittelu saatetaan nähdä asiana menneisyydestä, johon ei tarvi panostaa. Arkkitehtuurin nähdään rakentuvan asteittain ohjelmoinnin yhteydessä, iteraatioiden edetessä, refaktorointia hyödyntäen. Toisaalta perinteisen suunnitteluvetoisen ohjelmistokehityksen kannattajat ovat saattaneet nähdä ketterät menetelmät hieman amatöörimäisenä touhuna, joka soveltuu hyvin vain tietynlaisiin, tyypillisesti pieniin projekteihin. Artikkelin ja esimerkiksi saman aihepiirin kirjan Agile Software Architecture~\citep{babar2013agile} kantavana teemana on tälläisen vastakkainasettelun häivyttäminen - ketteryys ja arkkitehtuuri voivat elää yhteiseloa olla jopa toisiaan tukevia asioita.

Oikealla tavalla suunniteltu arkkitehtuuri voi olla hyvä työkalu parantamaan ohjelmiston laatua sekä vähentämään kehitykseen käytettyä aikaa ja kuluja, kuten myös ketterät menetelmät voidaan nähdä laatua, tuottavuutta, kannattavuutta ja ennen kaikkea sovelluskehittäjien tyytyväisyyttä lisäävänä asiana.~\citep{abrahamsson2010agility, babar2013agile}

\section*{Kuinka arkkitehtuurisuunnittelua sovelletaan ketterässä ohjelmistokehityksessä}
\thispagestyle{empty}

A systematic mapping study on the combination of software architecture and agile development~\citep{yang2016systematic} listaa 43 arkkitehtuurisuunnittelun lähestymistapaa sekä niihin liittyviä tutkimuksia. Listattuna ovat muun moassa iteratiivinen lähestymistapa sekä nollaiteraatio.

Iteratiivinen arkkitehtuuri (Emergent Architecture) sulauttaa arkkitehtuurisuunnittelun osaksi ketterän tuotannon iteraatoita. Arkkitehtuurisuunnittelua tehdään vain tarpeellinen määrä kyseisen iteraation tavoitteiden, eli käytännössä uusien käyttäjätoiminnallisuuksien toteuttamiseksi. Tässä mallissa arkkitehtuuri rakentuu refaktoroinnin kautta: sitä ei olla määritelty etukäteen vaan se kehittyy koodin mukana.~\citep{abrahamsson2010agility, babar2013agile}

Abrahamsson et al. (2010) artikkelissa todetaan, että arkkitehtuurin suunnittelu tulisi tehdä mahdollisimman aikaisin, sillä se kattaa merkittäviä päätöksiä järjestelmän rakenteesta ja käyttäytymisestä - näitä päätöksiä on vaikea kumota tai muuttaa myöhemmissä vaiheissa projektia.

Nollaiteraatiossa (Iteration Zero) on tarkoitus siirtää osa arkkitehtuurisuunnittelua tapahtuvaksi ennen varsinaista tuotantovaihetta olevaan iteraatioon sekä parantaa arkkitehtuuria tulevissa iteraatioissa~\citep{babar2013agile}. Tässä vaiheessa on mahdollista tehdä ns. walking skeleton, eli eräänlainen prototyyppi arkkitehtuurista, jonka tarkoitus on rakentua järjestelmän kasvaessa. Esimerkiksi Scrumiin on ollut ennen määriteltynä ns. pregame-vaihe, jossa korkean tason arkkitehtuuri on määritelty. Tämä on kuitenkin poistettu myöhemmistä Scrumin kuvauksista~\citep{babar2013agile}. 

Ketterässä kehityksessä muutoksen pitäisikin olla ns. tavoiteltavissa oleva asia, joten etukäteen tehtävää arkkitehtuurisuunnittelua ei pitäisi periaatteessa esiintyä. 

Babar et al. (2013) kirjassa viitatun tutkimuksen mukaan Scrum-ohjelmistokehityksessä kuitenkin yleisesti hyödynnetään nollasprinttejä, BDUF-henkistä suunnittelua ja erillisten arkkitehtuuritiimien käyttöä. Abrahamsson et al. (2010) mainitsee Davide Falessin tutkimuksen, jonka mukaan ketterää hyödyntävät ohjelmistokehittäjät yleisesti pitävät arkkitehtuuria merkityksellisenä apuvälineenä, joka muun muassa helpottaa kommunikointia ja suunnitteluvaihtoehtojen arviointia.

Ketterän manifestin~\citep{fowler2001agile} mukaan parhaat arkkitehtuurit syntyvät itsenäisten kehitystiimin sisällä. Tämän voi nähdä lisäävän arkkitehtuuriin sitoutumista ja vähentävän dokumentaatioita. Toisaalta varsinkin suuremmissa organisaatioissa tämä ideaali ei välttämättä voi olla mahdollinen - arkkitehtuurilliset päätökset voi esimerkiksi tulla tiimin ulkopuolelta. Tällöin taas esimerkiksi dokumentaation tärkeys voi korostua.

Se, miten arkkitehtuuri ja ketterä ohjelmistotuotanto yhdistetään, ei ole itsestäänselvyys. Tästä on onneksi tehty tutkimusta - hyviä lähtökohtia aihepiirin tutkimukseen vaikuttaisi olevan esimerkiksi lähteissä mainitut Yang et al. (2016) kartoitustutkimus ja Babar et al. (2013) aiheesta koottu kirja.


\thispagestyle{empty}
